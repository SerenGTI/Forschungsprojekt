\documentclass[conference,a4paper]{IEEEtran}

\usepackage[noadjust]{cite}
\usepackage[utf8]{inputenc}
%\usepackage{csquotes}
\usepackage{hyperref}
\usepackage{amsmath}

\newcommand{\todo}{{\color{red}TODO}}

\bibliography{literature.bib}
\title{Forschungsprojekt}
\author{\IEEEauthorblockN{Simon König}
\IEEEauthorblockA{(3344789)\\
st156571@stud.uni-stuttgart.de}
\and
\IEEEauthorblockN{Leon Matzner}
\IEEEauthorblockA{(3315161)\\
@stud.uni-stuttgart.de}
\and
\IEEEauthorblockN{Felix Rollbühler}
\IEEEauthorblockA{(3310069)\\
@stud.uni-stuttgart.de}
\and
\IEEEauthorblockN{Jakob Schmid}
\IEEEauthorblockA{()\\
@stud.uni-stuttgart.de}}
\date{\today}



\begin{document}

\maketitle


\begin{abstract}
In this paper we will analyze and compare the graph frameworks Galois, Ligra, Polymer, Gemini and Giraph in their parformance. All the frameworks will be tested in shared memory and Galois, Gemini and Giraph are tested on a distributed cluster as well.
Furthermore we will give some insight on the complexity of writing custom applications based on these frameworks.
\end{abstract}

\begin{IEEEkeywords}
graphs, distributed computing, Galois, Ligra, Polymer, Giraph, Gluon
\end{IEEEkeywords}



\section{Introduction}
%\todo

This paper makes the following contributions:
\begin{itemize}
  \item Comparison of some of the most widely used graph based calculation frameworks
  \item 
  \item 
  \item 
\end{itemize}

\section{Overview of the framworks}
\subsection{Galois and Gluon}
%!TEX root=main.tex

Galois \cite{Galois} is a general purpose library designed for parallel programming. The Galois system supports fine grain tasks, allows for autonomous, speculative execution of these tasks and grants control over the task scheduling policies to the application. It also simplifies the implementation of parallel applications by providing an implicitly parallel unordered-set iterator.\\
For graph analytics purposes a topology aware work stealing scheduler, a priority scheduler and a library of scalable data structures have been implemented. Galois includes applications for for many graph analytics problems, among these are single-source shorthest-paths (sssp) and pagerank. Both of these applications can be executed in shared memory systems and due to the Gluon integration in a distributed environment.\\
Gluon \cite{vertGalois} reduces the communication overhead needed in distributed systems for graph analysis by exploiting structural and temporal invariants.


\subsection{Ligra}
%!TEX root=../../main.tex

Ligra is a lightweight graph processing framework for shared memory machines \cite{Ligra}. It offers a vertex-centric programming interface which can be used to apply a function to each vertex or outgoing edge of a set of vertices in parallel. While doing so the framework can generate a set of active vertices for the next iteration. This abstraction makes it well suited for writing graph traversal algorithms.

When mapping over edges Ligra optimizes algorithms by switching between a push-based and a pull-based approach based on the size of the set. When the size of the set is above a threshold a pull approach is used. If the threshold is not reached a push approach is used.


\subsection{Polymer}
%!TEX root=main.tex

\subsection{Polymer}
Installing Polymer is straight forward.



\subsection{Gemini}
%!TEX root=../../main.tex



\subsection{Giraph}
%!TEX root=../../main.tex
Apache Giraph is an example for an open-source system similar to Pregel.
Thus, Giraph's computation model is closely related to the BSP model discussed in \autoref{sec:bsp}. 
This means that Giraph is based on computation units that communicate using messages and are synchonized with barriers \cite{Giraph}.

The input to a Giraph computation is always a directed graph. Not only the edges but also the vertices have a value attached to them. The graph topology is thus not only defined by the vertices and edges but also their initial values.
Furthermore, one can mutate the graph by adding or removing vertices and edges during computation.

The computation is vertex oriented and iterative.
For each iteration step called superstep, the \emph{compute} method implementing the algorithm is invoked on each active vertex, with every vertex being active in the beginning.
This method receives messages sent in the previous superstep as well as its vertex value and the values of outgoing edges.
With this data, the vertex values are modified and messages to other vertices are sent.
Communication between vertices is only performed via messages, so a vertex has no direct access to values of other vertices. The only visible information is the set of attached edges and their weights.
Supersteps are synchronized using barriers, meaning that all messages only get delivered in the following superstep and computation for the next superstep can only begin after every vertex has finished computing the current superstep.
Edge and vertex values are retained across supersteps.
Any vertex can stop computing (i.e. setting its state to inactive) at any time but incoming messages will reactivate the vertex.
A vote-to-halt method is applied, i.e. if all vertices are inactive or if a user defined superstep number is reached the computation ends.
Once calculation is finished, each vertex outputs some local information (e.g. the final vertex value) as result.

In order for Giraph to achieve scalability and parallelization, it is built on top of Apache Hadoop \cite{Giraph}.
Hadoop is a MapReduce infrastructure providing a fault tolerant basis for large scale data processing.
Hadoop supplies a distributed file system (HDFS), on which all computations are performed.
Giraph is thus, even when only using a single node, running in a distributed manner.
Hence, expanding single-node processing to a multi-node cluster is seamless.
Giraph uses the Map functionality of Hadoop to run the algorithms. Reduce is only used as the identity function.

Giraph being an Apache project makes it the most actively maintained and tested project in our comparison. While writing this paper, several new updates were pushed to Giraph's source repository\footnote{\url{https://gitbox.apache.org/repos/asf?p=giraph.git}}.


\section{And}
%!TEX root = ../main.tex

\subsection{An Overview of Important Graph Formats}
Since every frameworks uses different graph input formats, we supply a conversion tool capable of translating from EdgeList to the required formats.
Data Sets retreived from KONECT can be directly read and translated.

The following sections explain the output formats of our conversion tool.
\subsubsection{AdjacencyList}
The AdjacencyList and WeightedAdjacencyList formats are used by Ligra and Polymer. The format was initially specified for the Problem Based Benchmark Suite, an open source repository to compare different parallel programming methodologies in terms of performance and code quality \cite{pbbs}.

The file looks as follows
\begin{equation*}
	n, m, o_1, \ldots, o_n, t_1, \ldots, t_m
\end{equation*}
where commas are \texttt{\textbackslash n}. First, $n$ is the number of vertices and $m$ the number of edges in the graph.

The $o_k$ are the so-called offsets. Each vertex $k$ has an offset $o_k$, that describes an index in the following list of the $t_i$.
The $t_i$ are vertex IDs describing target nodes of a directed edge. 
The index $o_k$ in the list of target nodes is the point where edges outgoing from vertex $k$ begin to be declared. So vertex $k$ has the outgoing edges
\begin{equation*}
	(k, t_{o_k}), (k, t_{o_k+1}),\ldots, (k, t_{o_{k+1}-1}).
\end{equation*}

For the WeightedAdjacencyList format, the weights are appended to the end of the file in an order corresponding to the target nodes.

\subsubsection{EdgeList}
The EdgeList format is the most intuitive and one of the most commonly used in online data set repositories. The KONECT database uses this format and thus it is the input format for our conversion tool.

An edge list is a set of directed eges $(s_1,t_1),(s_2,t_2),\ldots$ where $s_i$ is a vertex ID representing the start vertex and $t_i$ is a vertex ID representing the target vertex.
In the format, there is one edges per line and the vertex IDs $s_i, t_i$ are separated with any whitespace character.

For a WeightedEdgeList, the edge weights are appended to each line, again separated by a whitespace character.

\subsubsection{Binary EdgeList}
The binary EdgeList format is used by Gemini.

For $s_i, t_i$ some vertex IDs and $w_i$ the weight of a directed edge $(s_i,t_i, w_i)$, Gemini requires the following input format
\begin{equation*}
	s_1t_1w_1s_2t_2w_2\ldots
\end{equation*}
where $s_i,t_i$ have \texttt{uint32} data type and the optional weights are \texttt{float32}.
Gemini will derive the number of edges from the file size, so there is no file header or anything similar allowed.

\subsubsection{Giraph's I/O formats}
Giraph is capable of parsing many different input and output formats. All of those are explained in Giraph's JavaDoc\footnote{\url{http://giraph.apache.org/apidocs/index.html}}.
Both edge- and vertex-centric input formats are possible.

One can even define their own input graph representation or output format. For the purposes of this paper, we used an existing format similar to AdjacencyList but represented in a JSON-like manner.

In this format, the vertex IDs are specified as \texttt{long} with \texttt{double} vertex values, \texttt{float} out-edge weights.
Each line in the graph file looks as follows
\begin{equation*}
	[s,v_s,[[t_1, w_{t_1}], [t_2, w_{t_2}]...]]
\end{equation*}
with $s$ being a vertex ID, $v_s$ the vertex value of vertex $s$. The values $t_i$ are vertices for which an edge from $s$ to $t_i$ exists. The directed edge $(s,t_i)$ has weight $w_{t_i}$.

There is no surrounding pair of brackets and no commas separating the lines as it would be expected in a JSON format.
%hier oder supplementary data?

\section{Testing methods}
%!TEX root=../main.tex


\subsection{Testing Methods}
The testing methods cover the test environment, including hardware and the setup of the frameworks, the graphs used, followed by the algorithms utilized and how the times are measured. 

\subsubsection{Environment}
For testing the graph processing systems, we used 5 machines with two AMD EPYC 7401 (24-Cores) and 256 GB of RAM each. One of those machines was only used as part of the distributed cluster, since it only has 128 GB of RAM.
All five machines were running Ubuntu 18.04.2 LTS.

The setup of each framework was performed according to our provided installation guides available in Appendix \ref{app:installationGuides}.
All benchmark cases were initiated by our benchmark script available in our repository.
All five frameworks are tested on a single server.
Galois, Gemini and Giraph were benchmarked in on the distributed 5-node cluster as well.
Since Galois supports this parameter, we ran multiple tests comparing Galois' performance with different thread counts on a single machine.
Furthermore, Galois is a framework capable of utilizing hugepages. We include an evaluation using those on the single node as well.
Unless mentioned otherwise, we always show results of each framework utilizing 96 threads (i.e. the maximum on our machines) for the single-node evaluation.
The complete benchmark log files and extracted raw results are available in our repository\footnote{\url{https://github.com/SerenGTI/Forschungsprojekt}}.


\subsubsection{Data Sets}
The graphs used in our testing can be seen in detail in \autoref{tbl:graphs}. We included a variaty of different graph sizes, from relatively small graphs like the flickr graph with 2 million edges up to an rMat28 with 4.2 billion edges. All graphs except the rMat27 and rMat28 are exemplary real-world graphs and were retrieved from the graph database\footnote{\url{http://konect.uni-koblenz.de/}} associated with the Koblenz Network Collection (KONECT)\cite{konect}.
Both the rMat27 and rMat28 were created with a modified version of a graph generator provided by Ligra (we changed the output format to EdgeList).
\begin{table}
	\centering
	\caption{Size Comparison of the Used Graphs}
	\begin{tabular}{crr}
		\hline
		\bf{Graph}&\# Vertices (M)&\# Edges (M)\\\hline
		flickr&    		0.1&  2\\
		orkut&          3&    117\\
		wikipedia&      12&   378\\
		twitter&     	52&   1963\\
		rMat27&         63&   2147\\
		friendster&     68&   2586\\
		rMat28&         121&  4294\\
		\hline
	\end{tabular}
	\label{tbl:graphs}
\end{table}

\subsubsection{Algorithms}
The three problems Breadth-first search (BFS), PageRank (PR) and Single-source shortest-path (SSSP) were used to benchmark each framework with every graph.
We always show the results of PageRank with a maximum of five iterations.
For frameworks that support multiple implementations (i.e. PageRank in push and pull modes), we included both in our evaluation.
We chose SSSP and BFS because they are iterative traversal algorithms. Active vertices typically are locally concentrated in the graph. The results of these algorithms can give some insight on the behaviour of the framework with other, similar behaving algorithms.
PageRank on the other hand is an algorithm that is very different to SSSP or BFS for that matter. With PR, there are many active vertices spread across the entire graph, enforcing different data handling strategies from the framework.

In detail, the algorithms for each framework are:
\todo{Algorithmen auflisten}
\begin{itemize}
	\item Ligra supports SSSP based on BellmanFord, BFS and two implementations of PageRank. The two implementations are a regular PR and a Delata Variant.
	\item Polymer supports the same algoritms as ligra.
	\item Gemini supports all of our tested algorithms and there are no setting options or specifications which implementations for the algorithms are used.
	\item Galois supports all of our tested algorithms too, with both a Push and a Pull variant for PageRank available. In the distributed scenario, there are Push and Pull versions for SSSP and BFS available as well. It also supports multiple implementations of the shared-memory allgorithms. The default implementation of SSSP is deltaTile. A lot of setting options are avilable as well, but we're gone with the defaults.
	\item Giraph does not natively supply a BFS algorithm, so in our comparisons a custom implementation is used. For SSSP, slight variations had to be made to the default implementation, to allow us to use different start vertices. For PageRank the supplied implementation is used.
\end{itemize}


\subsubsection{Measurements}
For every framework, we measured the \emph{execution time} as the time from start to finish of the console command.
For the \emph{calculation time}, we tried to extract only the time the framework actually executed the algorithm.
Furthermore, the \emph{overhead} is the time difference between execution time and calculation time. This includes time to read the input graph, initialization and any other tasks other than the actual user-defined algorithm.
Measuring the execution time is straight forward and was done using console time stamps.
For measuring the calculation time, we came up with the following:
\begin{itemize}
	\item For Galois, we extract console log time stamps. Galois outputs \enquote{\texttt{Reading graph complete.}}. Calculation time is the time from this output to the end of execution.

	This is not the most realiable way for measuring the calculation times.
	Not only due to unavoidable buffering in the console output we expect the measured time to be larger than the actual.
	First, it is not clear that all initialization is in fact complete after reading the graph. Second, we include time in the measurement that is used for cleanup after calculation.

	However, this method is the only way of retreiving any measurements without introducing custom modifications to the Galois source code.

	\item Polymer outputs the name of the algorithm followed by an internally measured time.

	\item Gemini outputs a line \texttt{exec\_time=x}, which was used to measure the calculation time.

	\item Ligra outputs its time measurement with \texttt{Running time : x}.

	\item Giraph has built in timers for the iterations (supersteps), the sum of those is the computation time.
\end{itemize}
Each evaluation consisting of graph, framework and algorithm was run 10 times, allowing us to smooth slight variations in the measured times.
Later on, we provide the mean values of the individual times as well as the standard deviation where meaningful.


Hier sollten wir auch evtl Vergleichsgrundlagen für Aufsetzen und Apps schreiben einführen, wenn wir das mit aufnehmen.


\section{Results}
\subsection{Komplexität Aufsetzen}
Systeme wie distr. Giraph bei denen man erstmal ewig suchen muss bis man überhaupt einen Guide findet sollten hier schlechter abschneiden. \emph{vorausgesetzt, wir nehmen das hier überhaupt mit auf..}

\subsection{Komplexität eigene Apps schreiben}
Schwierig objektiv zu vergleichen.


\subsection{Pure Performance-Ergebnisse}

\subsection{Ergebnisse Calc time}
der offensichtliche, wichtige Vergleich

\subsection{Ergebnisse exec time}
hier erhoffe ich mir einen Vergleich der Ladezeiten und erwarte, dass Systeme wie Giraph, die erstmal auf irgendwas warten schlecht abschneiden.
Aber vielleicht ist auch die setup time bei gleichen frameworks zwischen verteilt und shared memory ganz interessant zu vergleichen. 



\section{Discussion}




\section{Conclusion}
The conclusion goes here.



\section*{Acknowledgment}
We are using the graph frameworks Galois~\cite{Galois}, Ligra~\cite{Ligra}, Polymer~\cite{Polymer}, Gemini~\cite{Gemini} as well as Apache Giraph~\cite{Giraph}.

Also we use Gluon~\cite{vertGalois} for the distributed setups.

Gemini~\cite{Gemini}

\section*{Supplementary Data}\label{supplementaryData}
We have written a number of conversion tools and installation guides to help
users or develeopers with the use of the tested frameworks.

Everything can be retrieved on our GitHub repository
\url{http://www.github.com/serengti/Forschungsprojekt}.

For each Framework, there is a 


The graphs are downloaded from \cite{konect}.



% Beim Hinzufügen neuer Referenzen
% 1. die beiden Zeilen nach diesem Kommentar auskommentieren.
%    (gemeint ist \bibliographystyle... und \bibliography..)
%    Dann alle cache files im Ordner löschen.
% 2. Die folgenden Commands ausführen
% pdflatex main.tex
% bibtex main
% pdflatex main.tex 
%    das letzte wirft dann Fehler aber das stimmt so..
% 3. Jetzt main.bbl öffnen und den Inhalt hier drunter rein kopieren.
%    Man kann kein \input benutzen, weil main.bbl leer ist, wenn die beiden
%    Zeilen auskommentiert sind. Aber wenn sie nicht auskommentiert sind
%    kann kein PDF erzeugt werden.. Was zur Hölle denken die sich.
% 4. Die folgenden beiden Zeilen auskommentieren und pdflatex 
%    ausführen wie gewohnt.


% \bibliographystyle{IEEEtran}
% \bibliography{literature}


% Alles ab hier kommt aus main.bbl

% Generated by IEEEtran.bst, version: 1.12 (2007/01/11)
% Generated by IEEEtran.bst, version: 1.12 (2007/01/11)
% Generated by IEEEtran.bst, version: 1.12 (2007/01/11)
\begin{thebibliography}{1}
\providecommand{\url}[1]{#1}
\csname url@samestyle\endcsname
\providecommand{\newblock}{\relax}
\providecommand{\bibinfo}[2]{#2}
\providecommand{\BIBentrySTDinterwordspacing}{\spaceskip=0pt\relax}
\providecommand{\BIBentryALTinterwordstretchfactor}{4}
\providecommand{\BIBentryALTinterwordspacing}{\spaceskip=\fontdimen2\font plus
\BIBentryALTinterwordstretchfactor\fontdimen3\font minus
  \fontdimen4\font\relax}
\providecommand{\BIBforeignlanguage}[2]{{%
\expandafter\ifx\csname l@#1\endcsname\relax
\typeout{** WARNING: IEEEtran.bst: No hyphenation pattern has been}%
\typeout{** loaded for the language `#1'. Using the pattern for}%
\typeout{** the default language instead.}%
\else
\language=\csname l@#1\endcsname
\fi
#2}}
\providecommand{\BIBdecl}{\relax}
\BIBdecl

\bibitem{Galois}
\BIBentryALTinterwordspacing
D.~Nguyen, A.~Lenharth, and K.~Pingali, ``A lightweight infrastructure for
  graph analytics,'' in \emph{Proceedings of the Twenty-Fourth ACM Symposium on
  Operating Systems Principles}, ser. SOSP ’13.\hskip 1em plus 0.5em minus
  0.4em\relax New York, NY, USA: Association for Computing Machinery, 2013, p.
  456–471. [Online]. Available: \url{https://doi.org/10.1145/2517349.2522739}
\BIBentrySTDinterwordspacing

\bibitem{vertGalois}
\BIBentryALTinterwordspacing
R.~Dathathri, G.~Gill, L.~Hoang, H.-V. Dang, A.~Brooks, N.~Dryden, M.~Snir, and
  K.~Pingali, ``Gluon: A communication-optimizing substrate for distributed
  heterogeneous graph analytics,'' in \emph{Proceedings of the 39th ACM SIGPLAN
  Conference on Programming Language Design and Implementation}, ser. PLDI
  2018.\hskip 1em plus 0.5em minus 0.4em\relax New York, NY, USA: Association
  for Computing Machinery, 2018, p. 752–768. [Online]. Available:
  \url{https://doi.org/10.1145/3192366.3192404}
\BIBentrySTDinterwordspacing

\bibitem{Polymer}
\BIBentryALTinterwordspacing
K.~Zhang, R.~Chen, and H.~Chen, ``Numa-aware graph-structured analytics,'' in
  \emph{Proceedings of the 20th ACM SIGPLAN Symposium on Principles and
  Practice of Parallel Programming}, ser. PPoPP 2015.\hskip 1em plus 0.5em
  minus 0.4em\relax New York, NY, USA: Association for Computing Machinery,
  2015, p. 183–193. [Online]. Available:
  \url{https://doi.org/10.1145/2688500.2688507}
\BIBentrySTDinterwordspacing

\bibitem{adjListFormat}
\BIBentryALTinterwordspacing
J.~Shun, G.~Blelloch, J.~Fineman, P.~Gibbons, A.~Kyrola, K.~Tangwonsan, and
  H.~V. Simhadri. (2020, Jun.) {Problem Based Benchmark Suite}. graphIO.html.
  [Online]. Available: \url{http://www.cs.cmu.edu/~pbbs/benchmarks/}
\BIBentrySTDinterwordspacing

\bibitem{Ligra}
\BIBentryALTinterwordspacing
J.~Shun and G.~E. Blelloch, ``Ligra: A lightweight graph processing framework
  for shared memory,'' in \emph{Proceedings of the 18th ACM SIGPLAN Symposium
  on Principles and Practice of Parallel Programming}, ser. PPoPP ’13.\hskip
  1em plus 0.5em minus 0.4em\relax New York, NY, USA: Association for Computing
  Machinery, 2013, p. 135–146. [Online]. Available:
  \url{https://doi.org/10.1145/2442516.2442530}
\BIBentrySTDinterwordspacing

\bibitem{Gemini}
\BIBentryALTinterwordspacing
X.~Zhu, W.~Chen, W.~Zheng, and X.~Ma, ``Gemini: A computation-centric
  distributed graph processing system,'' in \emph{12th {USENIX} Symposium on
  Operating Systems Design and Implementation ({OSDI} 16)}.\hskip 1em plus
  0.5em minus 0.4em\relax Savannah, GA: {USENIX} Association, Nov. 2016, pp.
  301--316. [Online]. Available:
  \url{https://www.usenix.org/conference/osdi16/technical-sessions/presentation/zhu}
\BIBentrySTDinterwordspacing

\bibitem{Giraph}
\BIBentryALTinterwordspacing
A.~S. Foundation. (2020, Jun.) {Apache Giraph}. [Online]. Available:
  \url{https://giraph.apache.org}
\BIBentrySTDinterwordspacing

\bibitem{konect}
J.~Kunegis, ``Konect: the koblenz network collection,'' 05 2013, pp.
  1343--1350.

\end{thebibliography}


% Alles bis hier kommt aus main.bbl



\end{document}
