\documentclass[conference,a4paper]{IEEEtran}

\usepackage[noadjust]{cite}
\usepackage[utf8]{inputenc}
%\usepackage{csquotes}
\usepackage{hyperref}
\usepackage{amsmath}

\newcommand{\todo}{{\color{red}TODO}}

\bibliography{literature.bib}
\title{A Comparison of Graph Processing Systems}
\author{\IEEEauthorblockN{Simon König}
\IEEEauthorblockA{(3344789)\\
st156571@stud.uni-stuttgart.de}
\and
\IEEEauthorblockN{Leon Matzner}
\IEEEauthorblockA{(3315161)\\
@stud.uni-stuttgart.de}
\and
\IEEEauthorblockN{Felix Rollbühler}
\IEEEauthorblockA{(3310069)\\
@stud.uni-stuttgart.de}
\and
\IEEEauthorblockN{Jakob Schmid}
\IEEEauthorblockA{()\\
@stud.uni-stuttgart.de}}
\date{\today}



\begin{document}

\maketitle


\begin{abstract}
In this paper we will analyze and compare the graph frameworks Galois, Ligra, Polymer, Gemini and Giraph in their parformance. All the frameworks will be tested in shared memory and Galois, Gemini and Giraph are tested on a distributed cluster as well.
Furthermore we will give some insight on the complexity of writing custom applications based on these frameworks.
\end{abstract}

\begin{IEEEkeywords}
graphs, distributed computing, Galois, Ligra, Polymer, Giraph, Gluon, Gemini
\end{IEEEkeywords}



\section{Introduction}
%\todo

This paper makes the following contributions:
\begin{itemize}
  \item Comparison of some of the most widely used graph based calculation frameworks
  \item 
  \item 
  \item 
\end{itemize}

\section{Overview of the framworks}
\subsection{Galois and Gluon}
%!TEX root=main.tex


test


\subsection{Gemini}
%!TEX root=../../main.tex

Gemini\cite{Gemini} is a framework for parallel graph processing. In comparrison to the other NUMA-aware frameworks we discuss here Gemini does's not support shared memory parallel processing, it is completely message based using MPI. It seems to be very lightweight and has only the very basically needed functionallity implemented, which leeds to clear source code and small binarys. A basic API hides data and computation distribution details from the user writing applications. It comes with five algorithms already implemented included the three we are testing in this paper. While setting it up and even while benchmarking we had various problems with bugs in the source code, like non zero terminated strings and missing return statements. While debugging we forked the origional repository and made our changes there. DODO link repository.


\subsection{Giraph}
%!TEX root=../../main.tex
Apache Giraph is an example for an open-source system similar to Pregel.
Thus, Giraph's computation model is closely related to the BSP model discussed in \autoref{sec:bsp}. 
This means that Giraph is based on computation units that communicate using messages and are synchonized with barriers \cite{Giraph}.

The input to a Giraph computation is always a directed graph. Not only the edges but also the vertices have a value attached to them. The graph topology is thus not only defined by the vertices and edges but also their initial values.
Furthermore, one can mutate the graph by adding or removing vertices and edges during computation.

The computation is vertex oriented and iterative.
For each iteration step called superstep, the \emph{compute} method implementing the algorithm is invoked on each active vertex, with every vertex being active in the beginning.
This method receives messages sent in the previous superstep as well as its vertex value and the values of outgoing edges.
With this data, the vertex values are modified and messages to other vertices are sent.
Communication between vertices is only performed via messages, so a vertex has no direct access to values of other vertices. The only visible information is the set of attached edges and their weights.
Supersteps are synchronized using barriers, meaning that all messages only get delivered in the following superstep and computation for the next superstep can only begin after every vertex has finished computing the current superstep.
Edge and vertex values are retained across supersteps.
Any vertex can stop computing (i.e. setting its state to inactive) at any time but incoming messages will reactivate the vertex.
A vote-to-halt method is applied, i.e. if all vertices are inactive or if a user defined superstep number is reached the computation ends.
Once calculation is finished, each vertex outputs some local information (e.g. the final vertex value) as result.

In order for Giraph to achieve scalability and parallelization, it is built on top of Apache Hadoop \cite{Giraph}.
Hadoop is a MapReduce infrastructure providing a fault tolerant basis for large scale data processing.
Hadoop supplies a distributed file system (HDFS), on which all computations are performed.
Giraph is thus, even when only using a single computation node, running in a distributed manner.
Hence, expanding single-node processing to a multi-node cluster is seamless.
Giraph uses the Map functionality of Hadoop to run the algorithms. Reduce is only used as the identity function.

Giraph being an Apache project makes it the most actively maintained and tested project in our comparison. While writing this paper, several new updates were pushed to Giraph's source repository\footnote{\url{https://gitbox.apache.org/repos/asf?p=giraph.git}}.


\subsection{Ligra}
%!TEX root=../../main.tex

Ligra\cite{Ligra} is a lightweight parallel graph processing framework for shared memory machines. It offers a programming inteface that allows expressing graph traversal algorithms in a simple way.

Algorithms can use the \texttt{EdgeMap} to make computations based on edges or the \texttt{VertexMap} to make computations based on vertices. Those mappings can be applied only to a subset of vertices. Based on the size of the vertex subset the framework automatically switches between a sparse and a dense representation to optimize speed and memory.



\subsection{Polymer}
%!TEX root=../../main.tex

Polymer is very similar to Ligra, in fact Polymer inherits the programming interfaces \texttt{EdgeMap} and \texttt{VertexMap} from Ligra as its main interface.

Polymer is a vertex-centric framework, that tries to circumvent some of the random memory access drawbacks of such a design. It treats a NUMA machine as a distributed cluster and splits work and graph data accordingly between the nodes.\cite[p.187]{Polymer}
Application-defined data is not distributed. Other runtime state data is allocated in a distributed way but only accessed through a global lookup table.\cite[p.184]{Polymer}






\section{And}
%!TEX root = ./main.tex

\subsection{An overview of some graph formats}
A rather big portion of our time was invested in figuring out which graph framework requires which graph formats. We thus decided to give an overview over all the formats we encountered, with explanation on how they represent the graph.

Additionally, to make life in the future a little bit easier, we wrote multiple tools to convert graphs acquired from Snap or Konect to the required formats. Additional information on this is available in the section \nameref{supplementaryData} at the end.

\subsubsection{AdjacencyList}
The AdjacencyList and WeightedAdjacencyList formats\cite{adjListFormat} are used by Ligra and Polymer. They represent the directed edges of a graph as a number of offsets that point to a set of target nodes in the file.
First the file contains the number of vertices $n$ and edges $m$, followed by an offset for each vertex. This offset specifies at what point in the following list of numbers the information for a node begins.
Lastly the file format contains a list of target nodes.
The numbers are all separated by newlines.
\begin{gather*}
n\\
m\\
o_1\\
o_2\\
\vdots\\
o_n\\
t_1\\
t_2\\
\vdots\\
t_m
\end{gather*}
The offsets $o_i=k$ and $o_{i+1}=k+j$ mean that vertex $i$ has $j$ outgoing edges, these edges are
\begin{equation*}
	(i,t_k),(i,t_{k+1}),\ldots,(i,t_{k+j-1})
\end{equation*}

For the WeightedAdjacencyList format, the weights are just appended to the end of the file in the same order as the edges.

\subsubsection{EdgeList}
The EdgeList format is probably the easiest to understand and is one of the most commonly used in the online graph repositories. The directed eges $(s_1,t_1),(s_2,t_2),\ldots$ are represented in the following way.
\begin{align*}
	&s_1,t_1\\
	&s_2,t_2\\
	&\quad\vdots\\
	&s_m,t_m
\end{align*}

\subsubsection{Binary EdgeList}
The binary EdgeList format is used by Gemini. Finding information on this format required reverse engineering of the Gemini code.

We found that Gemini requires the following input format
\begin{equation*}
	s_1t_1w_1s_2t_2w_2\ldots
\end{equation*}
where $s_i,t_i$ have \texttt{uint32} ?? data type and the weights are \texttt{float32}. 
Gemini will derive the number of edges from the file size, so there is no file header or anything similar allowed.

\subsubsection{Giraph's numerous I/O formats}
%hier oder supplementary data?

\section{Testing methods}
%!TEX root=../main.tex

\subsection{Hardware and Software}
For testing all systems we used 5 machines with 96 CPU cores each (48 physical) and 256 GB of RAM. One of those machines only had 128 GB of RAM, this one was only used as part of the calculation cluster for the distributed systems.
All servers were running Ubuntu 19.x.
Setup of each framework was performed according to our provided installation guides.


\subsection{Benchmark setup}
We measured the total runtime of each process as well as the actual calculation times. The console output of each frameworks was used as an indicator for when loading the graph data was finished. This proved to be more difficult than expected in some cases.

Each test case consisting of graph, framework and in some cases even the algorithm was run 10 times, allowing us to smooth slight variations in the measured times. This also helps to reduce the error we introduced by measuring time through the console log.

All benchmarks were initiated by our benchmark script that is available in our repository.

We will compare both the computation times as well es the execution times. This way we compare how much setup time each framework has.


Hier sollten wir auch evtl Vergleichsgrundlagen für Aufsetzen und Apps schreiben einführen, wenn wir das mit aufnehmen.


\section{Results}
\subsection{Komplexität Aufsetzen}
Systeme wie distr. Galois, bei denen man erstmal ewig suchen muss bis man überhaupt einen Guide findet sollten hier schlechter abschneiden. \emph{vorausgesetzt, wir nehmen das hier überhaupt mit auf..}

\subsection{Komplexität eigene Apps schreiben}
Schwierig objektiv zu vergleichen.


\subsection{Pure Performance-Ergebnisse}

\subsection{Ergebnisse Calc time}
der offensichtliche, wichtige Vergleich

\subsection{Ergebnisse exec time}

hier erhoffe ich mir einen Vergleich der Ladezeiten und erwarte, dass Systeme wie Giraph, die erstmal auf irgendwas warten schlecht abschneiden.
Aber vielleicht ist auch die setup time bei gleichen frameworks zwischen verteilt und shared memory ganz interessant zu vergleichen. 



\section{Discussion}




\section{Conclusion}
The conclusion goes here.



\section*{Acknowledgment}
We are using the graph frameworks Galois~\cite{Galois}, Ligra~\cite{Ligra}, Polymer~\cite{Polymer}, Gemini~\cite{Gemini} as well as Apache Giraph~\cite{Giraph}.

Also we use Gluon~\cite{vertGalois} for the distributed Galois setups.

Gemini~\cite{Gemini}

%We would like to thank our supervisor Heiko Geppert for continued guidance and support.

\section*{Supplementary Data}\label{supplementaryData}
We have written a number of conversion tools and installation guides to help
users or develeopers with the use of the tested frameworks.

Our GitHub repository: \url{http://www.github.com/serengti/Forschungsprojekt}.

The graphs were downloaded from the graph database associated with the Koblenz Network Collection (KONECT)\cite{konect}.



% Beim Hinzufügen neuer Referenzen
% 1. die beiden Zeilen nach diesem Kommentar auskommentieren.
%    (gemeint ist \bibliographystyle... und \bibliography..)
%    Dann alle cache files im Ordner löschen.
% 2. Die folgenden Commands ausführen
% pdflatex main.tex
% bibtex main
% pdflatex main.tex 
%    das letzte wirft dann Fehler aber das stimmt so..
% 3. Jetzt main.bbl öffnen und den Inhalt hier drunter rein kopieren.
%    Man kann kein \input benutzen, weil main.bbl leer ist, wenn die beiden
%    Zeilen auskommentiert sind. Aber wenn sie nicht auskommentiert sind
%    kann kein PDF erzeugt werden.. Was zur Hölle denken die sich.
% 4. Die folgenden beiden Zeilen auskommentieren und pdflatex 
%    ausführen wie gewohnt.


% \bibliographystyle{IEEEtran}
% \bibliography{literature}


% Alles ab hier kommt aus main.bbl

% Generated by IEEEtran.bst, version: 1.12 (2007/01/11)
% Generated by IEEEtran.bst, version: 1.12 (2007/01/11)
% Generated by IEEEtran.bst, version: 1.12 (2007/01/11)
\begin{thebibliography}{1}
\providecommand{\url}[1]{#1}
\csname url@samestyle\endcsname
\providecommand{\newblock}{\relax}
\providecommand{\bibinfo}[2]{#2}
\providecommand{\BIBentrySTDinterwordspacing}{\spaceskip=0pt\relax}
\providecommand{\BIBentryALTinterwordstretchfactor}{4}
\providecommand{\BIBentryALTinterwordspacing}{\spaceskip=\fontdimen2\font plus
\BIBentryALTinterwordstretchfactor\fontdimen3\font minus
  \fontdimen4\font\relax}
\providecommand{\BIBforeignlanguage}[2]{{%
\expandafter\ifx\csname l@#1\endcsname\relax
\typeout{** WARNING: IEEEtran.bst: No hyphenation pattern has been}%
\typeout{** loaded for the language `#1'. Using the pattern for}%
\typeout{** the default language instead.}%
\else
\language=\csname l@#1\endcsname
\fi
#2}}
\providecommand{\BIBdecl}{\relax}
\BIBdecl

\bibitem{Galois}
\BIBentryALTinterwordspacing
D.~Nguyen, A.~Lenharth, and K.~Pingali, ``A lightweight infrastructure for
  graph analytics,'' in \emph{Proceedings of the Twenty-Fourth ACM Symposium on
  Operating Systems Principles}, ser. SOSP ’13.\hskip 1em plus 0.5em minus
  0.4em\relax New York, NY, USA: Association for Computing Machinery, 2013, p.
  456–471. [Online]. Available: \url{https://doi.org/10.1145/2517349.2522739}
\BIBentrySTDinterwordspacing

\bibitem{vertGalois}
\BIBentryALTinterwordspacing
R.~Dathathri, G.~Gill, L.~Hoang, H.-V. Dang, A.~Brooks, N.~Dryden, M.~Snir, and
  K.~Pingali, ``Gluon: A communication-optimizing substrate for distributed
  heterogeneous graph analytics,'' in \emph{Proceedings of the 39th ACM SIGPLAN
  Conference on Programming Language Design and Implementation}, ser. PLDI
  2018.\hskip 1em plus 0.5em minus 0.4em\relax New York, NY, USA: Association
  for Computing Machinery, 2018, p. 752–768. [Online]. Available:
  \url{https://doi.org/10.1145/3192366.3192404}
\BIBentrySTDinterwordspacing

\bibitem{Polymer}
\BIBentryALTinterwordspacing
K.~Zhang, R.~Chen, and H.~Chen, ``Numa-aware graph-structured analytics,'' in
  \emph{Proceedings of the 20th ACM SIGPLAN Symposium on Principles and
  Practice of Parallel Programming}, ser. PPoPP 2015.\hskip 1em plus 0.5em
  minus 0.4em\relax New York, NY, USA: Association for Computing Machinery,
  2015, p. 183–193. [Online]. Available:
  \url{https://doi.org/10.1145/2688500.2688507}
\BIBentrySTDinterwordspacing

\bibitem{adjListFormat}
\BIBentryALTinterwordspacing
J.~Shun, G.~Blelloch, J.~Fineman, P.~Gibbons, A.~Kyrola, K.~Tangwonsan, and
  H.~V. Simhadri. (2020, Jun.) {Problem Based Benchmark Suite}. graphIO.html.
  [Online]. Available: \url{http://www.cs.cmu.edu/~pbbs/benchmarks/}
\BIBentrySTDinterwordspacing

\bibitem{Ligra}
\BIBentryALTinterwordspacing
J.~Shun and G.~E. Blelloch, ``Ligra: A lightweight graph processing framework
  for shared memory,'' in \emph{Proceedings of the 18th ACM SIGPLAN Symposium
  on Principles and Practice of Parallel Programming}, ser. PPoPP ’13.\hskip
  1em plus 0.5em minus 0.4em\relax New York, NY, USA: Association for Computing
  Machinery, 2013, p. 135–146. [Online]. Available:
  \url{https://doi.org/10.1145/2442516.2442530}
\BIBentrySTDinterwordspacing

\bibitem{Gemini}
\BIBentryALTinterwordspacing
X.~Zhu, W.~Chen, W.~Zheng, and X.~Ma, ``Gemini: A computation-centric
  distributed graph processing system,'' in \emph{12th {USENIX} Symposium on
  Operating Systems Design and Implementation ({OSDI} 16)}.\hskip 1em plus
  0.5em minus 0.4em\relax Savannah, GA: {USENIX} Association, Nov. 2016, pp.
  301--316. [Online]. Available:
  \url{https://www.usenix.org/conference/osdi16/technical-sessions/presentation/zhu}
\BIBentrySTDinterwordspacing

\bibitem{Giraph}
\BIBentryALTinterwordspacing
A.~S. Foundation. (2020, Jun.) {Apache Giraph}. [Online]. Available:
  \url{https://giraph.apache.org}
\BIBentrySTDinterwordspacing

\bibitem{konect}
J.~Kunegis, ``Konect: the koblenz network collection,'' 05 2013, pp.
  1343--1350.

\end{thebibliography}


% Alles bis hier kommt aus main.bbl



\end{document}
