\documentclass[conference,a4paper]{IEEEtran}

\usepackage[noadjust]{cite}
\usepackage[utf8]{inputenc}
%\usepackage{csquotes}
\usepackage{hyperref}
\usepackage{amsmath}

\bibliography{literature.bib}
\title{Forschungsprojekt}
\author{Simon König (3344789) \and Leon Matzner (3315161) \and Felix Rollbühler (3310069) \and Jakob Schmid (??????)}
\date{\today}



\begin{document}

\maketitle


\begin{abstract}
In this paper we will analyze and compare the graph frameworks Galois, Ligra and Polymer in their parformance. We will also set up and compare these frameworks in a distributed scenario.
\end{abstract}


\begin{IEEEkeywords}
Galois, Ligra, Polymer, distributed computing, Gluon
\end{IEEEkeywords}

\section{Introduction}
My intro... blah blah \cite{Polymer}.




%!TEX root = ../main.tex

\subsection{An Overview of Important Graph Formats}
Since every frameworks uses different graph input formats, we supply a conversion tool capable of translating from EdgeList to the required formats.
Data Sets retreived from KONECT can be directly read and translated.

The following sections explain the output formats of our conversion tool.
\subsubsection{AdjacencyList}
The AdjacencyList and WeightedAdjacencyList formats are used by Ligra and Polymer. The format was initially specified for the Problem Based Benchmark Suite, an open source repository to compare different parallel programming methodologies in terms of performance and code quality \cite{pbbs}.

The file looks as follows
\begin{equation*}
	n, m, o_1, \ldots, o_n, t_1, \ldots, t_m
\end{equation*}
where commas are \texttt{\textbackslash n}. First, $n$ is the number of vertices and $m$ the number of edges in the graph.

The $o_k$ are the so-called offsets. Each vertex $k$ has an offset $o_k$, that describes an index in the following list of the $t_i$.
The $t_i$ are vertex IDs describing target nodes of a directed edge. 
The index $o_k$ in the list of target nodes is the point where edges outgoing from vertex $k$ begin to be declared. So vertex $k$ has the outgoing edges
\begin{equation*}
	(k, t_{o_k}), (k, t_{o_k+1}),\ldots, (k, t_{o_{k+1}-1}).
\end{equation*}

For the WeightedAdjacencyList format, the weights are appended to the end of the file in an order corresponding to the target nodes.

\subsubsection{EdgeList}
The EdgeList format is the most intuitive and one of the most commonly used in online data set repositories. The KONECT database uses this format and thus it is the input format for our conversion tool.

An edge list is a set of directed eges $(s_1,t_1),(s_2,t_2),\ldots$ where $s_i$ is a vertex ID representing the start vertex and $t_i$ is a vertex ID representing the target vertex.
In the format, there is one edges per line and the vertex IDs $s_i, t_i$ are separated with any whitespace character.

For a WeightedEdgeList, the edge weights are appended to each line, again separated by a whitespace character.

\subsubsection{Binary EdgeList}
The binary EdgeList format is used by Gemini.

For $s_i, t_i$ some vertex IDs and $w_i$ the weight of a directed edge $(s_i,t_i, w_i)$, Gemini requires the following input format
\begin{equation*}
	s_1t_1w_1s_2t_2w_2\ldots
\end{equation*}
where $s_i,t_i$ have \texttt{uint32} data type and the optional weights are \texttt{float32}.
Gemini will derive the number of edges from the file size, so there is no file header or anything similar allowed.

\subsubsection{Giraph's I/O formats}
Giraph is capable of parsing many different input and output formats. All of those are explained in Giraph's JavaDoc\footnote{\url{http://giraph.apache.org/apidocs/index.html}}.
Both edge- and vertex-centric input formats are possible.

One can even define their own input graph representation or output format. For the purposes of this paper, we used an existing format similar to AdjacencyList but represented in a JSON-like manner.

In this format, the vertex IDs are specified as \texttt{long} with \texttt{double} vertex values, \texttt{float} out-edge weights.
Each line in the graph file looks as follows
\begin{equation*}
	[s,v_s,[[t_1, w_{t_1}], [t_2, w_{t_2}]...]]
\end{equation*}
with $s$ being a vertex ID, $v_s$ the vertex value of vertex $s$. The values $t_i$ are vertices for which an edge from $s$ to $t_i$ exists. The directed edge $(s,t_i)$ has weight $w_{t_i}$.

There is no surrounding pair of brackets and no commas separating the lines as it would be expected in a JSON format.

\section{Things}
%!TEX root=main.tex

Galois \cite{Galois} is a general purpose library designed for parallel programming. The Galois system supports fine grain tasks, allows for autonomous, speculative execution of these tasks and grants control over the task scheduling policies to the application. It also simplifies the implementation of parallel applications by providing an implicitly parallel unordered-set iterator.\\
For graph analytics purposes a topology aware work stealing scheduler, a priority scheduler and a library of scalable data structures have been implemented. Galois includes applications for for many graph analytics problems, among these are single-source shorthest-paths (sssp) and pagerank. Both of these applications can be executed in shared memory systems and due to the Gluon integration in a distributed environment.\\
Gluon \cite{vertGalois} reduces the communication overhead needed in distributed systems for graph analysis by exploiting structural and temporal invariants.


%!TEX root=main.tex

\subsection{Polymer}
Installing Polymer is straight forward.




\section{Conclusion}
The conclusion goes here.



\section*{Acknowledgment}
We are using the graph frameworks Galois~\cite{Galois}, Ligra~\cite{Ligra} and Polymer~\cite{Polymer}.

Also we use Gluon~\cite{vertGalois} for the distributed setups.




Gemini~\cite{Gemini}








% Das hier ist mega beschissen: 
% Beim Hinzufügen neuer Referenzen
% 1. die beiden Zeilen nach diesem Kommentar auskommentieren.
%    (gemeint ist \bibliographystyle... und \bibliography..)
%    Dann alle cache files im Ordner löschen.
% 2. Die folgenden Commands ausführen
% pdflatex main.tex
% bibtex main
% pdflatex main.tex 
%    das letzte wirft dann Fehler aber das stimmt so..
% 3. Jetzt main.bbl öffnen und den Inhalt hier drunter rein kopieren.
%    Man kann kein \input benutzen, weil main.bbl leer ist, wenn die beiden
%    Zeilen auskommentiert sind. Aber wenn sie nicht auskommentiert sind
%    kann kein PDF erzeugt werden.. Was zur Hölle denken die sich.
% 4. Die folgenden beiden Zeilen auskommentieren und pdflatex 
%    ausführen wie gewohnt.


% \bibliographystyle{IEEEtran}
% \bibliography{literature}


% Alles ab hier kommt aus main.bbl

% Generated by IEEEtran.bst, version: 1.12 (2007/01/11)
% Generated by IEEEtran.bst, version: 1.12 (2007/01/11)
\begin{thebibliography}{1}
\providecommand{\url}[1]{#1}
\csname url@samestyle\endcsname
\providecommand{\newblock}{\relax}
\providecommand{\bibinfo}[2]{#2}
\providecommand{\BIBentrySTDinterwordspacing}{\spaceskip=0pt\relax}
\providecommand{\BIBentryALTinterwordstretchfactor}{4}
\providecommand{\BIBentryALTinterwordspacing}{\spaceskip=\fontdimen2\font plus
\BIBentryALTinterwordstretchfactor\fontdimen3\font minus
  \fontdimen4\font\relax}
\providecommand{\BIBforeignlanguage}[2]{{%
\expandafter\ifx\csname l@#1\endcsname\relax
\typeout{** WARNING: IEEEtran.bst: No hyphenation pattern has been}%
\typeout{** loaded for the language `#1'. Using the pattern for}%
\typeout{** the default language instead.}%
\else
\language=\csname l@#1\endcsname
\fi
#2}}
\providecommand{\BIBdecl}{\relax}
\BIBdecl

\bibitem{Polymer}
\BIBentryALTinterwordspacing
K.~Zhang, R.~Chen, and H.~Chen, ``Numa-aware graph-structured analytics,'' in
  \emph{Proceedings of the 20th ACM SIGPLAN Symposium on Principles and
  Practice of Parallel Programming}, ser. PPoPP 2015.\hskip 1em plus 0.5em
  minus 0.4em\relax New York, NY, USA: Association for Computing Machinery,
  2015, p. 183–193. [Online]. Available:
  \url{https://doi.org/10.1145/2688500.2688507}
\BIBentrySTDinterwordspacing

\bibitem{adjListFormat}
\BIBentryALTinterwordspacing
J.~Shun, G.~Blelloch, J.~Fineman, P.~Gibbons, A.~Kyrola, K.~Tangwonsan, and
  H.~V. Simhadri. (2020, Jun.) {Problem Based Benchmark Suite}. graphIO.html.
  [Online]. Available: \url{http://www.cs.cmu.edu/~pbbs/benchmarks/}
\BIBentrySTDinterwordspacing

\bibitem{Galois}
\BIBentryALTinterwordspacing
D.~Nguyen, A.~Lenharth, and K.~Pingali, ``A lightweight infrastructure for
  graph analytics,'' in \emph{Proceedings of the Twenty-Fourth ACM Symposium on
  Operating Systems Principles}, ser. SOSP ’13.\hskip 1em plus 0.5em minus
  0.4em\relax New York, NY, USA: Association for Computing Machinery, 2013, p.
  456–471. [Online]. Available: \url{https://doi.org/10.1145/2517349.2522739}
\BIBentrySTDinterwordspacing

\bibitem{Ligra}
\BIBentryALTinterwordspacing
J.~Shun and G.~E. Blelloch, ``Ligra: A lightweight graph processing framework
  for shared memory,'' in \emph{Proceedings of the 18th ACM SIGPLAN Symposium
  on Principles and Practice of Parallel Programming}, ser. PPoPP ’13.\hskip
  1em plus 0.5em minus 0.4em\relax New York, NY, USA: Association for Computing
  Machinery, 2013, p. 135–146. [Online]. Available:
  \url{https://doi.org/10.1145/2442516.2442530}
\BIBentrySTDinterwordspacing

\bibitem{vertGalois}
\BIBentryALTinterwordspacing
R.~Dathathri, G.~Gill, L.~Hoang, H.-V. Dang, A.~Brooks, N.~Dryden, M.~Snir, and
  K.~Pingali, ``Gluon: A communication-optimizing substrate for distributed
  heterogeneous graph analytics,'' in \emph{Proceedings of the 39th ACM SIGPLAN
  Conference on Programming Language Design and Implementation}, ser. PLDI
  2018.\hskip 1em plus 0.5em minus 0.4em\relax New York, NY, USA: Association
  for Computing Machinery, 2018, p. 752–768. [Online]. Available:
  \url{https://doi.org/10.1145/3192366.3192404}
\BIBentrySTDinterwordspacing

\bibitem{Gemini}
\BIBentryALTinterwordspacing
X.~Zhu, W.~Chen, W.~Zheng, and X.~Ma, ``Gemini: A computation-centric
  distributed graph processing system,'' in \emph{12th {USENIX} Symposium on
  Operating Systems Design and Implementation ({OSDI} 16)}.\hskip 1em plus
  0.5em minus 0.4em\relax Savannah, GA: {USENIX} Association, Nov. 2016, pp.
  301--316. [Online]. Available:
  \url{https://www.usenix.org/conference/osdi16/technical-sessions/presentation/zhu}
\BIBentrySTDinterwordspacing

\end{thebibliography}



% Alles bis hier kommt aus main.bbl

\end{document}
