%!TEX root = ../main.tex

\subsection{An overview of some graph formats}
A rather big portion of our time was invested in figuring out which graph framework requires which graph formats. We thus decided to give an overview over all the formats we encountered, with explanation on how they represent the graph.

Additionally, to make life in the future a little bit easier, we wrote multiple tools to convert graphs acquired from Snap or Konect to the required formats. Additional information on this is available in the section \nameref{supplementaryData} at the end.

\subsubsection{AdjacencyList}
The AdjacencyList and WeightedAdjacencyList formats\cite{adjListFormat} are used by Ligra and Polymer. They represent the directed edges of a graph as a number of offsets that point to a set of target nodes in the file.
First the file contains the number of vertices $n$ and edges $m$, followed by an offset for each vertex. This offset specifies at what point in the following list of numbers the information for a node begins.
Lastly the file format contains a list of target nodes.
The numbers are all separated by newlines.
\begin{gather*}%with hacked line spacing to save some space..
n\\[-1\jot]
m\\[-1\jot]
o_1\\[-1\jot]
o_2\\[-1\jot]
\vdots\\[-1\jot]
o_n\\[-1\jot]
t_1\\[-1\jot]
t_2\\[-1\jot]
\vdots\\[-1\jot]
t_m
\end{gather*}
The offsets $o_i=k$ and $o_{i+1}=k+j$ mean that vertex $i$ has $j$ outgoing edges, these edges are
\begin{equation*}
	(i,t_k),(i,t_{k+1}),\ldots,(i,t_{k+j-1})
\end{equation*}

For the WeightedAdjacencyList format, the weights are just appended to the end of the file in the same order as the edges.

\subsubsection{EdgeList}
The EdgeList format is probably the easiest to understand and is one of the most commonly used in the online graph repositories. The directed eges $(s_1,t_1),(s_2,t_2),\ldots$ or $(s_1,t_1, w_1),(s_2,t_2, w_2),\ldots$ are represented in the following way.
\begin{align*}
  &s_1\ \ t_1\ \ w_1\\
  &s_2\ \ t_2\ \ w_m\\
	&\quad\vdots\\
  &s_m\ \ t_m\ \ w_m
\end{align*}
The weights are optional, everything is ASCI encoded and the inline delimiter is a variable amount of any whitespace.

\subsubsection{Binary EdgeList}
The binary EdgeList format is used by Gemini. Finding information on this format required reverse engineering of the Gemini code.

We found that Gemini requires the following input format
\begin{equation*}
	s_1t_1w_1s_2t_2w_2\ldots
\end{equation*}
where $s_i,t_i$ have \texttt{uint32} data type and the optional weights are \texttt{float32}.
Gemini will derive the number of edges from the file size, so there is no file header or anything similar allowed.

\subsubsection{Giraph's I/O formats}
Giraph is capable of parsing many different input and output formats. All of those are explained in Giraph's JavaDoc.
Both edge- and vertex-centric input formats are possible.

One can even define their own input graph representation or output format. For the purposes of this paper, we used the existing \texttt{JsonLongDoubleFloatDoubleVertexInputFormat}\cite{GiraphInputFormat}.
Our graph conversion tool outputs this format.

In this format, the vertex IDs are specified as \texttt{long} with \texttt{double} vertex values, \texttt{float} out-edge weights and the messages are of type \texttt{double} (all are Java data types).
All this is in a JSON-like format, so each line in the graph file looks as follows
\begin{equation*}
	[s,v_s,[[t_1, w_{t_1}], [t_2, w_{t_2}]...]]
\end{equation*}
with $s$ being a vertex ID, $v_s$ the vertex value of vertex $s$. The values $t_i$ are vertices for which an edge from $s$ to $t_i$ exists. The directed edge $(s,t_i)$ has weight $w_{t_i}$.

There is surrounding pair of brackets and no commas separating the lines as it would be expected in a JSON format.
