%!TEX root=../../main.tex


\subsubsection{PageRank}
In this section we compare the PageRank performance of the various frameworks. As usual we begin with the single-node performance and finish by discussing the distributed variants.

\paragraph{Single-Node}
Ligra and Polymer support both regular and Delta-PageRank variants.
Ligra's regular PR implementation is faster on 4 of 7 graphs. If the regular version is slower than delta, that is only by a small difference. Explicitly, regular is slower than delta by a range of 6\% to 19\%\ on twitter, rMat27 or friendster. For the other graphs, the delta version is slower by a far greater margin of 13\% to 68\%. Because we only run five iterations of PR, the optimizations of the delta algorithm over the regular implementation are not yet visible.
Hence, we only show the results of Ligra's regular PageRank implementation in our evaluation.
Polymer's delta implementation on the other hand can use the optimizations on a much smaller number of iterations to its advantage. For Polymer we found the delta version to be faster on all graphs except rMat28. Delta-PR is on average 15\%\ faster on the first six graphs, while only being 0.3\%\ slower on rMat28. 
Thus, the following only shows Polymer's faster Delta-PR implementation.
Giraph required more than the available 250 GB of RAM on any graph larger than wikipedia, hence all of Giraph's results for the larger graphs are missing here.
\begin{figure*}[ht]
	\hfil
	\begin{subfigure}{0.32\textwidth}
		\includegraphics[width=\linewidth]{../../plots/singleNodePR_calcTime.png}
		\caption{Calculation time}
		\label{fig:singleNodePR_calc}
	\end{subfigure}
	\hfil
	\begin{subfigure}{0.32\textwidth}
		\includegraphics[width=\linewidth]{../../plots/singleNodePR_execTime.png}
		\caption{Execution time}
		\label{fig:singleNodePR_exec}
	\end{subfigure}
	\hfil
	\begin{subfigure}{0.32\textwidth}
		\includegraphics[width=\linewidth]{../../plots/singleNodePR_overheadTIme.png}
		\caption{Overhead}
		\label{fig:singleNodePR_overhead}
	\end{subfigure}
	\hfil
	\caption{Average times for PR on a single computation node}
	\label{fig:singleNodePR}
\end{figure*}
\begin{figure*}[h]
	\hfil
	\begin{subfigure}{0.32\textwidth}
		\includegraphics[width=\linewidth]{../../plots/distributedPR_calcTime.png}
		\caption{Calculation time}
		\label{fig:distributedPR_calc}
	\end{subfigure}
	\hfil
	\begin{subfigure}{0.32\textwidth}
		\includegraphics[width=\linewidth]{../../plots/distributedPR_execTime.png}
		\caption{Execution time}
		\label{fig:distributedPR_exec}
	\end{subfigure}
	\hfil
	\begin{subfigure}{0.32\textwidth}
		\includegraphics[width=\linewidth]{../../plots/distributedPR_overheadTime.png}
		\caption{Overhead}
		\label{fig:distributedPR_overhead}
	\end{subfigure}
	\hfil
	\caption{Average times for PR on the distributed cluster}
	\label{fig:distributedPR}
\end{figure*}

The calculation times show some odd behaviour of Galois Push. The required time is less than 1ms, regardless of the graph (cf. \autoref{fig:singleNodePR_calc}). Meanwhile there was no output produced, that would indicate any kind of error. These results would make the calculation times of Galois Push the smallest on all graphs, with a difference of at least one order of magnitude. However, we are very suspicious of these results and thus exclude the calculation time for Galois Push in further comparisons. 
We believe that the output we used for our measurements contains an error,
because the \emph{execution} time of Galois Push is always considerably longer than that of Galois Pull (cf. \autoref{fig:singleNodePR_exec}).

For the three graphs on which Giraph computed successfully, it is the slowest framework in both calculation and execution times. Further, it is the slowest by three orders of magnitude in the calculation time and one to two orders of magnitude in the execution time (cf \autoref{fig:singleNodePR}). On the larger graphs (i.e. those, where there is no data for Giraph), Gemini and Ligra are always slowest in execution time.
Contrary to this, Galois Pull has the smallest execution times on all graphs except flickr (cf. \autoref{fig:singleNodePR_exec}). Ligra is fastest on flickr, while being second fastest on wikipedia, rMat27 and rMat28. Galois Push is second fastest on orkut and friendster.
Interestingly, the execution time for Ligra is at a maximum for twitter. The required time is steadily decreasing with increasing graph size.
\todo{Begründung?}



\paragraph{Distributed}
Our benchmark results of PageRank on the distributed cluster can be seen in \autoref{fig:distributedPR}.
First of all, Giraph was unable to complete the test on rMat28 because it ran out of memory, thus this result is missing.
When comparing the calculation times in \autoref{fig:distributedPR_calc} to the execution times in \autoref{fig:distributedPR_exec}, we see similar behaviour of all frameworks.
This means that unlike with SSSP or BFS, the calculation times and execution times are similar with respect to the relations of the frameworks to one another.

Gemini has the shortest calculation times on all graphs (cf. \autoref{fig:distributedPR_calc}).
The two Galois implementations are second on flickr, orkut, friendster and rMat28, with Giraph being second on the others.
Generally, the calculation of Galois Push is anywhere from 6\% (flickr) to 46\% (orkut) faster than the Pull counterpart.

This applies to the execution times in almost the same way (cf. \autoref{fig:distributedPR_exec}).
Gemini is the fastest on all graphs except orkut, where both Galois implementations are faster. Galois Pull takes 13s, whereas Gemini requires 19.4s on orkut.
Again, as for the calculation times, Galois is the second fastest framework on flickr, wikipedia, friendster and rMat28.
And Galois Push has smaller execution times than the Pull version because the overhead times for both implementations are similar (cf. \autoref{fig:distributedPR_overhead}).

