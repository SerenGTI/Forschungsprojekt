%!TEX root=../../main.tex

Ligra is a lightweight graph processing framework for shared memory machines \cite{Ligra}. 
It offers a vertex-centric programming interface which consists of two routines. 
The \texttt{VertexMap} applies a user defined function to each vertex and \texttt{EdgeMap} applies a user defined function to each outgoing edge of a set of vertices. 
Active vertices can be represented by a \texttt{VertexSubset}. 
This data structure can be passed to both of the routines and is maintained by them. 
That makes Ligra well suited for expressing graph traversal algorithms.

When \texttt{EdgeMap} is used Ligra decides between a push and a pull style execution.
The deciding factor is the number of active vertices and their outdegree.
When this number is small a push style execution is used, otherwise a pull style execution is used.
The default threshold is set to one twentieth of all edges.
This hybrid approach can increase performance significantly \cite{hybridBFS}.
When push and pull mode changes the representation of the \texttt{VertexSubset} is also changed. A push style execution corresponds to a sparse representation as array of the active vertex IDs, while a pull style execution corresponds to a dense representation as bitmap.
