%!TEX root=../../main.tex
%TODO add bibliography for origGalois and cite it here
Galois \cite{origGalois} is a general purpose library designed for parallel programming. The system reduces the complexity of writing parallel applications by providing implicitly parallel (unordered or partially ordered) set iterators. These iterators perform operations optimistically, detect arising conflicts and resolve these by invoking inverse methods accordingly. The tasks in the sets can be ordered, the tasks may be executed out of order, but the conflict resolution will ensure semantic equivalence to sequential strictly ordered execution.

%Galois \cite{Galois} is a general purpose library designed for parallel programming. The Galois system supports fine grain tasks, allows for autonomous, speculative execution of these tasks and grants control over the task scheduling policies to the application. It also simplifies the implementation of parallel applications by providing an implicitly parallel unordered-set iterators.

The graph analysis subsystem of Galois \cite{Galois} provides a library of scalable data structures and a topology aware priority scheduler, including optimizations for distributed execution. The scheduler splits the tasks into bags according to a specified partial order, which in turn provide the cores with chunks of tasks. A global map manages the various bags. Every thread keeps a lazy cache of a portion of the global map, in order to reduce the strain on the global map. Galois includes applications for many graph analytics problems, among these are SSSP, BFS and pagerank. For most of these applications Galois offers several different algorithms to perform these analytics problems and many options e.g. the amount of threads used or policies for splitting the graph. All of these applications can be executed in shared memory systems and, due to the Gluon integration, with a few modifications in a distributed environment

%For graph analytics purposes a topology aware work stealing scheduler, a priority scheduler and a library of scalable data structures have been implemented. Galois includes applications for many graph analytics problems, among these are single-source shorthest-paths (sssp), breath-first-search (bfs) and pagerank. For most of these applications Galois offers several different algorithms to perform these analytics problems and many setting options like the amount of threads used or policies for splitting the graph. All of these applications can be executed in shared memory systems and, due to the Gluon integration, with a few modifications in a distributed environment.

Gluon \cite{vertGalois} is a framework written for Galois as a middleware for distributed graph analysis applications. It reduces the communication overhead needed in distributed environments by exploiting structural and temporal invariants. Depending on the used graph partitioning policy only a subset of the messages of a naive approach must be sent (structural invariant). Gluon establishes a mapping of local vertex ID's to the order in which the values will be sent/received between the owner and each mirror. A message includes a bit vector where a one in the i-th position means that the according vertex of the established order has been updated. Thus a message only has to include updated values without the need to state the vertex ID (temporal invariants). Gluon is embedded in Galois, but can be integrated in other graph analysis frameworks as well \cite{vertGalois}.

%The code of Gluon is embedded in Galois. It is possible to integrate Gluon in other frameworks \cite{vertGalois}.
