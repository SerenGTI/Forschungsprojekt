%!TEX root=../../main.tex


Apache Giraph\cite{Giraph} is an iterative graph processing framework, built on top of Apache Hadoop. Because of this, expanding shared memory processing to a multi-node cluster is almost seamless.

Giraph is inspired by the Bulk Synchronous Parallel model of distributed computation introduced by Leslie Valiant\cite{BSPforGiraph}. Meaning that it is based on computation units that communicate using messages and are synchonized with barriers.

The input to a Giraph computation is always a directed graph. Not only the edges but also the vertices have a value attached to them. The graph topology is thus not only defined by the vertices and edges but also their initial values.
Furthermore, one can mutate the graph by adding or removing vertices and edges during computation.

Computation is vertex oriented and iterative.
For each iteration step called superstep, the Compute method implementing the algorithm is invoked on each active vertex, with every vertex being active in the beginning.
This method receives messages sent in the previous superstep as well as its vertex value and the values of outgoing edges.
With this data the values are modified and messages to other vertices are sent.
Communication between vertices is only performed via messages, so a vertex has no access to values of other vertices or edges other than its own outgoing ones.

Supersteps are synchronized with barriers, meaning that all messages only get delivered in the following superstep and computation for the next superstep can only begin after every vertex has finished computing the current superstep.
Edge and vertex values are retained across supersteps.

Any vertex can stop computing (i.e. setting its state to inactive) at any time but incoming messages will make the vertex active again.
To end computation, a vote-to-halt method is applied. Each vertex outputs some local information (e.g. the final vertex value) as result.

Giraph is next to Galois probably the most actively developed framework in our comparison.
