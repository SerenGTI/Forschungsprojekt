%!TEX root=../../main.tex

Gemini is a framework for distributed graph processing \cite{Gemini}.
It was developed with the goal to deliver a generally better performance through efficient communication.
While most other graph processing systems achieve very good results in the shared-memory environment, they often deliver unsatisfactory results in distributed computing.
Furthermore, a well optimized single-threaded implementation often outperforms a distributed system \cite{scalability}.
Therefore it is necessary to not only focus on the performance of the computation but also of the performance of the communication.
Gemini tries to bridge the gap between efficient shared-memory and scalable distributed systems \cite{Gemini}.
To achieve this goal, Gemini, in contrast to the other frameworks discussed here, does not support shared-memory calculation, but chooses the distributed message-based approach from scratch.

The real bottleneck of distributed systems is not the communication itself, but the extra instructions, as well as memory references and a lower usage of multiple cores compared to the shared memory counterparts.
There are four main reasons for this.
The first reason is the use of hash maps to convert the vertex IDs between the global and the local state.
The second reason is the maintenance of vertex replicas on the different systems.
Another reason is the communication-bound apply phase in GAS abstraction.
And the last reason is the lack of dynamic scheduling.

Gemini tries to work around all the problems, by implemening a message-based system from scratch and getting rid of the extra mapping layer between shared-memory computation and communication.
Therefore Ligra's push-pull computation model was adopted and applied to the distributed computation.
Furthermore a chunk-based partitioning scheme was implemented, which allows to partition a graph without a large overhead.
Gemini also implements a co-scheduling mechanism to connect the computation and inter-node communication.

Gemini is fairly lightweight and has a clearly defined API between the core framework and the implementations of the individual algorithms. The five already implemented algorithms are Single-Source SSSP, BFS, PR, Connected-Components (CC) and Betweenness-Centrality (BC).

