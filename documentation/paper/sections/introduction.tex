%!TEX root=../main.tex
\section{Introduction}

In recent years, graph processing has become increasingly important, both in academic and industrial environments \cite{Gemini}.
In many areas, such as machine learning and data mining, but also scientific computations, problems can be modeled and solved with the help of graphs \cite{Polymer}.
Thus a new application domain called graph alnalytics was created.
Some business models are based on working with graphs, such as marketing analysis, advertising, or even Google's search engine.
In addition, the size of graphs has grown significantly in the last few years and will most likely continue to grow.
This makes the performance and memory efficency of graph analysis applications more important than ever.
But also the parallelism and distribution of graph algorithms becomes more important with increasing graph sizes.

Some frameworks were developed for this purpose.
One of the best known is the closed source distributed graph processing system Pregel \cite{pregel}.
Pregel, as well as some open source versions of it, like Giraph \cite{Giraph} and GraphX \cite{graphx}, was built to handle large graphs reliable and fault tolerant on large MapReduce infrastructures.
Due to the lack of cache locality, as these systems are written object oriented, they provide rather suboptimal performance.
Because of this, some non-uniform memory access (NUMA) aware systems like Ligra \cite{Ligra}, Polymer \cite{Polymer}, Galois \cite{Galois} and Gemini \cite{Gemini} were introduced.
\\
- Prolem der Vergleichbarkeit der Systeme\\
- Systeme entwickeln sich weiter und wurden nur zu deren Start verglichen\\
- Vergleiche wurden von Machern der Systeme erledigt\\
\\
3. Was wir zur Lösung des Problems beitragen\\
- ersten, die unabhängig alle Systeme Vergleichen 2020\\
- Wir vergleichen die Frameworks nach folgenden Gesichtspunkten:\\
(Benutzerfreundlichkeit, Performanz/Varianz, Weiterentwicklung)
- We compare several non-uniform memory access (NUMA) aware systems in terms of their performance on three graph algorithms (PageRank, SSSP, BFS).\\
- The comparison is performed on both real world data sets and synthetic graphs.
To provide a comparison to a non-NUMA aware system, Giraph\cite{Giraph} is included in the testing. Giraph itself has often been compared to other state of the art systems like Pregel or GraphX.
- Comparison of several state-of-the-art graph processing frameworks\\
\\



This paper makes the following contributions:
\begin{itemize}
  \item Comparison of several state-of-the-art graph processing frameworks
  \item
  \item
  \item
\end{itemize}


4. Wie ist das Paper aufgebaut\\
- Section 2: Relatet Work (Wie haben die Macher der Systeme die Systeme verglichen)\\
- Section 3: Preliminaries (Was muss man an technischen Details wissen)\\
- Section 4: Framework Overview (Welche Frameworks und was macht diese aus)\\
- Section 5: Evaluaten (Was wurde gemessen?, Wie wurde gemessen?, Was sind die Ergebnisse?)\\
- Section 6: Discussion (Warum sehen die Ergebnisse so aus, wie sie aussehen?)\\
- Section 7: Concluseion (Was lernen wir daraus?)\\
