%!TEX root=../main.tex

\section{Discussion}
\todo{unfertig}

There are two main cases that should be regarded in a comparison of the frameworks. One, the case of individual calculations on a graph, i.e. for each calculation, the graph has to be loaded. This is called the \emph{research} case. For research, frameworks with small execution times and small overhead are preferred. Case two on the other hand is the case of a running system, that performs multiple calculations on a single graph, without the need of reloading graph data with every calculation. This is the \emph{production} case.
In production, frameworks with short calculation times should be preferred because the overhead time is only spent once on startup.

Orthogonal to this, the graph size and topology of the system is a deciding factor. Large graphs often require a distributed cluster, because the graph is simply too large to fit in the RAM of just one machine. We observed this for example with Giraph on multiple cases.
Furthermore, some environments require a distributed framework setup regardless of any other factors.


%### SSSP
%### BFS
For algorithms that perform a traversal through the graph, with a concentrated and relativly small amount of active vertices, our results for SSSP and BFS are most relevant.






%### general
For graphs that are not too large for single node RAM and when performing research work, Galois as the fastest single-node framework is recommended.
It provides excellent performance, being orders of magnitude faster than the competing frameworks.
In some cases, especially on PageRank-like applications, tests to determine whether a push- or pull-style implementation is most efficient might be necessary.
Because hugepages provide an improvement in performance when running calculations on many different graphs, using them in research systems is beneficial as well.




%## production
On production systems we expect it to be possible to determine what configuration is best, before putting the system in service.
There are multiple possibilities to be accounted for

hp on/off
thread count.. weniger wichtig











%### PR
There is still some testing to be performed, in order to find the reason for the performance anomalies we encountered. 




%### Distributed
A distributed setup can have many advantages,


first, there is some data security, because many distributed file systems also support automatic data replication etc.



There is an argument to be made for systems like Hadoop.
A user defining an application does not need to know the topology of the system that runs the application later on.
Hadoop transparently splits the input data and distributes work among the worker nodes.
And Hadoop automatically handles node failures, a node failure does not immediately result in loss of data or a faulty computation result.
This is especially important on very large clusters, that can not easily be monitored by humans.
Furthermore, Giraph using hadoop shows a bias towards larger graphs, we observed an improvement in performance over single-node Giraph, when increasing the size of the input graph.
This shows the hadoop infrastructure to be working well. 
The Giraph/hadoop system is recommended for very large graphs on very large computation clusters.
Only a system like hadoop can reasonably be used on such large clusters.
Any of the other framewors will just result in large administration overhead due to node failures or other problems of distributed systems in general.




Especially for frameworks like Giraph, that require large amounts of memory, the distributed scenario allows them to complete their computations. 

